\documentclass[11pt]{article}

    \usepackage[breakable]{tcolorbox}
    \usepackage{parskip} % Stop auto-indenting (to mimic markdown behaviour)
    
    \usepackage[utf8]{inputenc}
    %\usepackage{ucs}
    \usepackage[T2A,T1]{fontenc}
    \usepackage[english,russian]{babel}
    \usepackage{textcase} 
    \usepackage{cmap}

    % Basic figure setup, for now with no caption control since it's done
    % automatically by Pandoc (which extracts ![](path) syntax from Markdown).
    \usepackage{graphicx}
    % Maintain compatibility with old templates. Remove in nbconvert 6.0
    \let\Oldincludegraphics\includegraphics
    % Ensure that by default, figures have no caption (until we provide a
    % proper Figure object with a Caption API and a way to capture that
    % in the conversion process - todo).
    \usepackage{caption}
    \DeclareCaptionFormat{nocaption}{}
    \captionsetup{format=nocaption,aboveskip=0pt,belowskip=0pt}

    \usepackage{float}
    \floatplacement{figure}{H} % forces figures to be placed at the correct location
    \usepackage{xcolor} % Allow colors to be defined
    \usepackage{enumerate} % Needed for markdown enumerations to work
    \usepackage{geometry} % Used to adjust the document margins
    \usepackage{amsmath} % Equations
    \usepackage{amssymb} % Equations
    \usepackage{textcomp} % defines textquotesingle
    % Hack from http://tex.stackexchange.com/a/47451/13684:
    \AtBeginDocument{%
        \def\PYZsq{\textquotesingle}% Upright quotes in Pygmentized code
    }
    \usepackage{upquote} % Upright quotes for verbatim code
    \usepackage{eurosym} % defines \euro
    \usepackage[mathletters]{ucs} % Extended unicode (utf-8) support
    \usepackage{fancyvrb} % verbatim replacement that allows latex
    \usepackage{grffile} % extends the file name processing of package graphics 
                         % to support a larger range
    \makeatletter % fix for old versions of grffile with XeLaTeX
    \@ifpackagelater{grffile}{2019/11/01}
    {
      % Do nothing on new versions
    }
    {
      \def\Gread@@xetex#1{%
        \IfFileExists{"\Gin@base".bb}%
        {\Gread@eps{\Gin@base.bb}}%
        {\Gread@@xetex@aux#1}%
      }
    }
    \makeatother
    \usepackage[Export]{adjustbox} % Used to constrain images to a maximum size
    \adjustboxset{max size={0.9\linewidth}{0.9\paperheight}}

    % The hyperref package gives us a pdf with properly built
    % internal navigation ('pdf bookmarks' for the table of contents,
    % internal cross-reference links, web links for URLs, etc.)
    \usepackage{hyperref}
    % The default LaTeX title has an obnoxious amount of whitespace. By default,
    % titling removes some of it. It also provides customization options.
    \usepackage{titling}
    \usepackage{longtable} % longtable support required by pandoc >1.10
    \usepackage{booktabs}  % table support for pandoc > 1.12.2
    \usepackage[inline]{enumitem} % IRkernel/repr support (it uses the enumerate* environment)
    \usepackage[normalem]{ulem} % ulem is needed to support strikethroughs (\sout)
                                % normalem makes italics be italics, not underlines
    \usepackage{mathrsfs}
    

    
    % Colors for the hyperref package
    \definecolor{urlcolor}{rgb}{0,.145,.698}
    \definecolor{linkcolor}{rgb}{.71,0.21,0.01}
    \definecolor{citecolor}{rgb}{.12,.54,.11}

    % ANSI colors
    \definecolor{ansi-black}{HTML}{3E424D}
    \definecolor{ansi-black-intense}{HTML}{282C36}
    \definecolor{ansi-red}{HTML}{E75C58}
    \definecolor{ansi-red-intense}{HTML}{B22B31}
    \definecolor{ansi-green}{HTML}{00A250}
    \definecolor{ansi-green-intense}{HTML}{007427}
    \definecolor{ansi-yellow}{HTML}{DDB62B}
    \definecolor{ansi-yellow-intense}{HTML}{B27D12}
    \definecolor{ansi-blue}{HTML}{208FFB}
    \definecolor{ansi-blue-intense}{HTML}{0065CA}
    \definecolor{ansi-magenta}{HTML}{D160C4}
    \definecolor{ansi-magenta-intense}{HTML}{A03196}
    \definecolor{ansi-cyan}{HTML}{60C6C8}
    \definecolor{ansi-cyan-intense}{HTML}{258F8F}
    \definecolor{ansi-white}{HTML}{C5C1B4}
    \definecolor{ansi-white-intense}{HTML}{A1A6B2}
    \definecolor{ansi-default-inverse-fg}{HTML}{FFFFFF}
    \definecolor{ansi-default-inverse-bg}{HTML}{000000}

    % common color for the border for error outputs.
    \definecolor{outerrorbackground}{HTML}{FFDFDF}

    % commands and environments needed by pandoc snippets
    % extracted from the output of `pandoc -s`
    \providecommand{\tightlist}{%
      \setlength{\itemsep}{0pt}\setlength{\parskip}{0pt}}
    \DefineVerbatimEnvironment{Highlighting}{Verbatim}{commandchars=\\\{\}}
    % Add ',fontsize=\small' for more characters per line
    \newenvironment{Shaded}{}{}
    \newcommand{\KeywordTok}[1]{\textcolor[rgb]{0.00,0.44,0.13}{\textbf{{#1}}}}
    \newcommand{\DataTypeTok}[1]{\textcolor[rgb]{0.56,0.13,0.00}{{#1}}}
    \newcommand{\DecValTok}[1]{\textcolor[rgb]{0.25,0.63,0.44}{{#1}}}
    \newcommand{\BaseNTok}[1]{\textcolor[rgb]{0.25,0.63,0.44}{{#1}}}
    \newcommand{\FloatTok}[1]{\textcolor[rgb]{0.25,0.63,0.44}{{#1}}}
    \newcommand{\CharTok}[1]{\textcolor[rgb]{0.25,0.44,0.63}{{#1}}}
    \newcommand{\StringTok}[1]{\textcolor[rgb]{0.25,0.44,0.63}{{#1}}}
    \newcommand{\CommentTok}[1]{\textcolor[rgb]{0.38,0.63,0.69}{\textit{{#1}}}}
    \newcommand{\OtherTok}[1]{\textcolor[rgb]{0.00,0.44,0.13}{{#1}}}
    \newcommand{\AlertTok}[1]{\textcolor[rgb]{1.00,0.00,0.00}{\textbf{{#1}}}}
    \newcommand{\FunctionTok}[1]{\textcolor[rgb]{0.02,0.16,0.49}{{#1}}}
    \newcommand{\RegionMarkerTok}[1]{{#1}}
    \newcommand{\ErrorTok}[1]{\textcolor[rgb]{1.00,0.00,0.00}{\textbf{{#1}}}}
    \newcommand{\NormalTok}[1]{{#1}}
    
    % Additional commands for more recent versions of Pandoc
    \newcommand{\ConstantTok}[1]{\textcolor[rgb]{0.53,0.00,0.00}{{#1}}}
    \newcommand{\SpecialCharTok}[1]{\textcolor[rgb]{0.25,0.44,0.63}{{#1}}}
    \newcommand{\VerbatimStringTok}[1]{\textcolor[rgb]{0.25,0.44,0.63}{{#1}}}
    \newcommand{\SpecialStringTok}[1]{\textcolor[rgb]{0.73,0.40,0.53}{{#1}}}
    \newcommand{\ImportTok}[1]{{#1}}
    \newcommand{\DocumentationTok}[1]{\textcolor[rgb]{0.73,0.13,0.13}{\textit{{#1}}}}
    \newcommand{\AnnotationTok}[1]{\textcolor[rgb]{0.38,0.63,0.69}{\textbf{\textit{{#1}}}}}
    \newcommand{\CommentVarTok}[1]{\textcolor[rgb]{0.38,0.63,0.69}{\textbf{\textit{{#1}}}}}
    \newcommand{\VariableTok}[1]{\textcolor[rgb]{0.10,0.09,0.49}{{#1}}}
    \newcommand{\ControlFlowTok}[1]{\textcolor[rgb]{0.00,0.44,0.13}{\textbf{{#1}}}}
    \newcommand{\OperatorTok}[1]{\textcolor[rgb]{0.40,0.40,0.40}{{#1}}}
    \newcommand{\BuiltInTok}[1]{{#1}}
    \newcommand{\ExtensionTok}[1]{{#1}}
    \newcommand{\PreprocessorTok}[1]{\textcolor[rgb]{0.74,0.48,0.00}{{#1}}}
    \newcommand{\AttributeTok}[1]{\textcolor[rgb]{0.49,0.56,0.16}{{#1}}}
    \newcommand{\InformationTok}[1]{\textcolor[rgb]{0.38,0.63,0.69}{\textbf{\textit{{#1}}}}}
    \newcommand{\WarningTok}[1]{\textcolor[rgb]{0.38,0.63,0.69}{\textbf{\textit{{#1}}}}}
    
    
    % Define a nice break command that doesn't care if a line doesn't already
    % exist.
    \def\br{\hspace*{\fill} \\* }
    % Math Jax compatibility definitions
    \def\gt{>}
    \def\lt{<}
    \let\Oldtex\TeX
    \let\Oldlatex\LaTeX
    \renewcommand{\TeX}{\textrm{\Oldtex}}
    \renewcommand{\LaTeX}{\textrm{\Oldlatex}}
    % Document parameters
    % Document title
    \title{Rahmatullaev 27}
    
    
    
    
    
% Pygments definitions
\makeatletter
\def\PY@reset{\let\PY@it=\relax \let\PY@bf=\relax%
    \let\PY@ul=\relax \let\PY@tc=\relax%
    \let\PY@bc=\relax \let\PY@ff=\relax}
\def\PY@tok#1{\csname PY@tok@#1\endcsname}
\def\PY@toks#1+{\ifx\relax#1\empty\else%
    \PY@tok{#1}\expandafter\PY@toks\fi}
\def\PY@do#1{\PY@bc{\PY@tc{\PY@ul{%
    \PY@it{\PY@bf{\PY@ff{#1}}}}}}}
\def\PY#1#2{\PY@reset\PY@toks#1+\relax+\PY@do{#2}}

\@namedef{PY@tok@w}{\def\PY@tc##1{\textcolor[rgb]{0.73,0.73,0.73}{##1}}}
\@namedef{PY@tok@c}{\let\PY@it=\textit\def\PY@tc##1{\textcolor[rgb]{0.25,0.50,0.50}{##1}}}
\@namedef{PY@tok@cp}{\def\PY@tc##1{\textcolor[rgb]{0.74,0.48,0.00}{##1}}}
\@namedef{PY@tok@k}{\let\PY@bf=\textbf\def\PY@tc##1{\textcolor[rgb]{0.00,0.50,0.00}{##1}}}
\@namedef{PY@tok@kp}{\def\PY@tc##1{\textcolor[rgb]{0.00,0.50,0.00}{##1}}}
\@namedef{PY@tok@kt}{\def\PY@tc##1{\textcolor[rgb]{0.69,0.00,0.25}{##1}}}
\@namedef{PY@tok@o}{\def\PY@tc##1{\textcolor[rgb]{0.40,0.40,0.40}{##1}}}
\@namedef{PY@tok@ow}{\let\PY@bf=\textbf\def\PY@tc##1{\textcolor[rgb]{0.67,0.13,1.00}{##1}}}
\@namedef{PY@tok@nb}{\def\PY@tc##1{\textcolor[rgb]{0.00,0.50,0.00}{##1}}}
\@namedef{PY@tok@nf}{\def\PY@tc##1{\textcolor[rgb]{0.00,0.00,1.00}{##1}}}
\@namedef{PY@tok@nc}{\let\PY@bf=\textbf\def\PY@tc##1{\textcolor[rgb]{0.00,0.00,1.00}{##1}}}
\@namedef{PY@tok@nn}{\let\PY@bf=\textbf\def\PY@tc##1{\textcolor[rgb]{0.00,0.00,1.00}{##1}}}
\@namedef{PY@tok@ne}{\let\PY@bf=\textbf\def\PY@tc##1{\textcolor[rgb]{0.82,0.25,0.23}{##1}}}
\@namedef{PY@tok@nv}{\def\PY@tc##1{\textcolor[rgb]{0.10,0.09,0.49}{##1}}}
\@namedef{PY@tok@no}{\def\PY@tc##1{\textcolor[rgb]{0.53,0.00,0.00}{##1}}}
\@namedef{PY@tok@nl}{\def\PY@tc##1{\textcolor[rgb]{0.63,0.63,0.00}{##1}}}
\@namedef{PY@tok@ni}{\let\PY@bf=\textbf\def\PY@tc##1{\textcolor[rgb]{0.60,0.60,0.60}{##1}}}
\@namedef{PY@tok@na}{\def\PY@tc##1{\textcolor[rgb]{0.49,0.56,0.16}{##1}}}
\@namedef{PY@tok@nt}{\let\PY@bf=\textbf\def\PY@tc##1{\textcolor[rgb]{0.00,0.50,0.00}{##1}}}
\@namedef{PY@tok@nd}{\def\PY@tc##1{\textcolor[rgb]{0.67,0.13,1.00}{##1}}}
\@namedef{PY@tok@s}{\def\PY@tc##1{\textcolor[rgb]{0.73,0.13,0.13}{##1}}}
\@namedef{PY@tok@sd}{\let\PY@it=\textit\def\PY@tc##1{\textcolor[rgb]{0.73,0.13,0.13}{##1}}}
\@namedef{PY@tok@si}{\let\PY@bf=\textbf\def\PY@tc##1{\textcolor[rgb]{0.73,0.40,0.53}{##1}}}
\@namedef{PY@tok@se}{\let\PY@bf=\textbf\def\PY@tc##1{\textcolor[rgb]{0.73,0.40,0.13}{##1}}}
\@namedef{PY@tok@sr}{\def\PY@tc##1{\textcolor[rgb]{0.73,0.40,0.53}{##1}}}
\@namedef{PY@tok@ss}{\def\PY@tc##1{\textcolor[rgb]{0.10,0.09,0.49}{##1}}}
\@namedef{PY@tok@sx}{\def\PY@tc##1{\textcolor[rgb]{0.00,0.50,0.00}{##1}}}
\@namedef{PY@tok@m}{\def\PY@tc##1{\textcolor[rgb]{0.40,0.40,0.40}{##1}}}
\@namedef{PY@tok@gh}{\let\PY@bf=\textbf\def\PY@tc##1{\textcolor[rgb]{0.00,0.00,0.50}{##1}}}
\@namedef{PY@tok@gu}{\let\PY@bf=\textbf\def\PY@tc##1{\textcolor[rgb]{0.50,0.00,0.50}{##1}}}
\@namedef{PY@tok@gd}{\def\PY@tc##1{\textcolor[rgb]{0.63,0.00,0.00}{##1}}}
\@namedef{PY@tok@gi}{\def\PY@tc##1{\textcolor[rgb]{0.00,0.63,0.00}{##1}}}
\@namedef{PY@tok@gr}{\def\PY@tc##1{\textcolor[rgb]{1.00,0.00,0.00}{##1}}}
\@namedef{PY@tok@ge}{\let\PY@it=\textit}
\@namedef{PY@tok@gs}{\let\PY@bf=\textbf}
\@namedef{PY@tok@gp}{\let\PY@bf=\textbf\def\PY@tc##1{\textcolor[rgb]{0.00,0.00,0.50}{##1}}}
\@namedef{PY@tok@go}{\def\PY@tc##1{\textcolor[rgb]{0.53,0.53,0.53}{##1}}}
\@namedef{PY@tok@gt}{\def\PY@tc##1{\textcolor[rgb]{0.00,0.27,0.87}{##1}}}
\@namedef{PY@tok@err}{\def\PY@bc##1{{\setlength{\fboxsep}{\string -\fboxrule}\fcolorbox[rgb]{1.00,0.00,0.00}{1,1,1}{\strut ##1}}}}
\@namedef{PY@tok@kc}{\let\PY@bf=\textbf\def\PY@tc##1{\textcolor[rgb]{0.00,0.50,0.00}{##1}}}
\@namedef{PY@tok@kd}{\let\PY@bf=\textbf\def\PY@tc##1{\textcolor[rgb]{0.00,0.50,0.00}{##1}}}
\@namedef{PY@tok@kn}{\let\PY@bf=\textbf\def\PY@tc##1{\textcolor[rgb]{0.00,0.50,0.00}{##1}}}
\@namedef{PY@tok@kr}{\let\PY@bf=\textbf\def\PY@tc##1{\textcolor[rgb]{0.00,0.50,0.00}{##1}}}
\@namedef{PY@tok@bp}{\def\PY@tc##1{\textcolor[rgb]{0.00,0.50,0.00}{##1}}}
\@namedef{PY@tok@fm}{\def\PY@tc##1{\textcolor[rgb]{0.00,0.00,1.00}{##1}}}
\@namedef{PY@tok@vc}{\def\PY@tc##1{\textcolor[rgb]{0.10,0.09,0.49}{##1}}}
\@namedef{PY@tok@vg}{\def\PY@tc##1{\textcolor[rgb]{0.10,0.09,0.49}{##1}}}
\@namedef{PY@tok@vi}{\def\PY@tc##1{\textcolor[rgb]{0.10,0.09,0.49}{##1}}}
\@namedef{PY@tok@vm}{\def\PY@tc##1{\textcolor[rgb]{0.10,0.09,0.49}{##1}}}
\@namedef{PY@tok@sa}{\def\PY@tc##1{\textcolor[rgb]{0.73,0.13,0.13}{##1}}}
\@namedef{PY@tok@sb}{\def\PY@tc##1{\textcolor[rgb]{0.73,0.13,0.13}{##1}}}
\@namedef{PY@tok@sc}{\def\PY@tc##1{\textcolor[rgb]{0.73,0.13,0.13}{##1}}}
\@namedef{PY@tok@dl}{\def\PY@tc##1{\textcolor[rgb]{0.73,0.13,0.13}{##1}}}
\@namedef{PY@tok@s2}{\def\PY@tc##1{\textcolor[rgb]{0.73,0.13,0.13}{##1}}}
\@namedef{PY@tok@sh}{\def\PY@tc##1{\textcolor[rgb]{0.73,0.13,0.13}{##1}}}
\@namedef{PY@tok@s1}{\def\PY@tc##1{\textcolor[rgb]{0.73,0.13,0.13}{##1}}}
\@namedef{PY@tok@mb}{\def\PY@tc##1{\textcolor[rgb]{0.40,0.40,0.40}{##1}}}
\@namedef{PY@tok@mf}{\def\PY@tc##1{\textcolor[rgb]{0.40,0.40,0.40}{##1}}}
\@namedef{PY@tok@mh}{\def\PY@tc##1{\textcolor[rgb]{0.40,0.40,0.40}{##1}}}
\@namedef{PY@tok@mi}{\def\PY@tc##1{\textcolor[rgb]{0.40,0.40,0.40}{##1}}}
\@namedef{PY@tok@il}{\def\PY@tc##1{\textcolor[rgb]{0.40,0.40,0.40}{##1}}}
\@namedef{PY@tok@mo}{\def\PY@tc##1{\textcolor[rgb]{0.40,0.40,0.40}{##1}}}
\@namedef{PY@tok@ch}{\let\PY@it=\textit\def\PY@tc##1{\textcolor[rgb]{0.25,0.50,0.50}{##1}}}
\@namedef{PY@tok@cm}{\let\PY@it=\textit\def\PY@tc##1{\textcolor[rgb]{0.25,0.50,0.50}{##1}}}
\@namedef{PY@tok@cpf}{\let\PY@it=\textit\def\PY@tc##1{\textcolor[rgb]{0.25,0.50,0.50}{##1}}}
\@namedef{PY@tok@c1}{\let\PY@it=\textit\def\PY@tc##1{\textcolor[rgb]{0.25,0.50,0.50}{##1}}}
\@namedef{PY@tok@cs}{\let\PY@it=\textit\def\PY@tc##1{\textcolor[rgb]{0.25,0.50,0.50}{##1}}}

\def\PYZbs{\char`\\}
\def\PYZus{\char`\_}
\def\PYZob{\char`\{}
\def\PYZcb{\char`\}}
\def\PYZca{\char`\^}
\def\PYZam{\char`\&}
\def\PYZlt{\char`\<}
\def\PYZgt{\char`\>}
\def\PYZsh{\char`\#}
\def\PYZpc{\char`\%}
\def\PYZdl{\char`\$}
\def\PYZhy{\char`\-}
\def\PYZsq{\char`\'}
\def\PYZdq{\char`\"}
\def\PYZti{\char`\~}
% for compatibility with earlier versions
\def\PYZat{@}
\def\PYZlb{[}
\def\PYZrb{]}
\makeatother


    % For linebreaks inside Verbatim environment from package fancyvrb. 
    \makeatletter
        \newbox\Wrappedcontinuationbox 
        \newbox\Wrappedvisiblespacebox 
        \newcommand*\Wrappedvisiblespace {\textcolor{red}{\textvisiblespace}} 
        \newcommand*\Wrappedcontinuationsymbol {\textcolor{red}{\llap{\tiny$\m@th\hookrightarrow$}}} 
        \newcommand*\Wrappedcontinuationindent {3ex } 
        \newcommand*\Wrappedafterbreak {\kern\Wrappedcontinuationindent\copy\Wrappedcontinuationbox} 
        % Take advantage of the already applied Pygments mark-up to insert 
        % potential linebreaks for TeX processing. 
        %        {, <, #, %, $, ' and ": go to next line. 
        %        _, }, ^, &, >, - and ~: stay at end of broken line. 
        % Use of \textquotesingle for straight quote. 
        \newcommand*\Wrappedbreaksatspecials {% 
            \def\PYGZus{\discretionary{\char`\_}{\Wrappedafterbreak}{\char`\_}}% 
            \def\PYGZob{\discretionary{}{\Wrappedafterbreak\char`\{}{\char`\{}}% 
            \def\PYGZcb{\discretionary{\char`\}}{\Wrappedafterbreak}{\char`\}}}% 
            \def\PYGZca{\discretionary{\char`\^}{\Wrappedafterbreak}{\char`\^}}% 
            \def\PYGZam{\discretionary{\char`\&}{\Wrappedafterbreak}{\char`\&}}% 
            \def\PYGZlt{\discretionary{}{\Wrappedafterbreak\char`\<}{\char`\<}}% 
            \def\PYGZgt{\discretionary{\char`\>}{\Wrappedafterbreak}{\char`\>}}% 
            \def\PYGZsh{\discretionary{}{\Wrappedafterbreak\char`\#}{\char`\#}}% 
            \def\PYGZpc{\discretionary{}{\Wrappedafterbreak\char`\%}{\char`\%}}% 
            \def\PYGZdl{\discretionary{}{\Wrappedafterbreak\char`\$}{\char`\$}}% 
            \def\PYGZhy{\discretionary{\char`\-}{\Wrappedafterbreak}{\char`\-}}% 
            \def\PYGZsq{\discretionary{}{\Wrappedafterbreak\textquotesingle}{\textquotesingle}}% 
            \def\PYGZdq{\discretionary{}{\Wrappedafterbreak\char`\"}{\char`\"}}% 
            \def\PYGZti{\discretionary{\char`\~}{\Wrappedafterbreak}{\char`\~}}% 
        } 
        % Some characters . , ; ? ! / are not pygmentized. 
        % This macro makes them "active" and they will insert potential linebreaks 
        \newcommand*\Wrappedbreaksatpunct {% 
            \lccode`\~`\.\lowercase{\def~}{\discretionary{\hbox{\char`\.}}{\Wrappedafterbreak}{\hbox{\char`\.}}}% 
            \lccode`\~`\,\lowercase{\def~}{\discretionary{\hbox{\char`\,}}{\Wrappedafterbreak}{\hbox{\char`\,}}}% 
            \lccode`\~`\;\lowercase{\def~}{\discretionary{\hbox{\char`\;}}{\Wrappedafterbreak}{\hbox{\char`\;}}}% 
            \lccode`\~`\:\lowercase{\def~}{\discretionary{\hbox{\char`\:}}{\Wrappedafterbreak}{\hbox{\char`\:}}}% 
            \lccode`\~`\?\lowercase{\def~}{\discretionary{\hbox{\char`\?}}{\Wrappedafterbreak}{\hbox{\char`\?}}}% 
            \lccode`\~`\!\lowercase{\def~}{\discretionary{\hbox{\char`\!}}{\Wrappedafterbreak}{\hbox{\char`\!}}}% 
            \lccode`\~`\/\lowercase{\def~}{\discretionary{\hbox{\char`\/}}{\Wrappedafterbreak}{\hbox{\char`\/}}}% 
            \catcode`\.\active
            \catcode`\,\active 
            \catcode`\;\active
            \catcode`\:\active
            \catcode`\?\active
            \catcode`\!\active
            \catcode`\/\active 
            \lccode`\~`\~ 	
        }
    \makeatother

    \let\OriginalVerbatim=\Verbatim
    \makeatletter
    \renewcommand{\Verbatim}[1][1]{%
        %\parskip\z@skip
        \sbox\Wrappedcontinuationbox {\Wrappedcontinuationsymbol}%
        \sbox\Wrappedvisiblespacebox {\FV@SetupFont\Wrappedvisiblespace}%
        \def\FancyVerbFormatLine ##1{\hsize\linewidth
            \vtop{\raggedright\hyphenpenalty\z@\exhyphenpenalty\z@
                \doublehyphendemerits\z@\finalhyphendemerits\z@
                \strut ##1\strut}%
        }%
        % If the linebreak is at a space, the latter will be displayed as visible
        % space at end of first line, and a continuation symbol starts next line.
        % Stretch/shrink are however usually zero for typewriter font.
        \def\FV@Space {%
            \nobreak\hskip\z@ plus\fontdimen3\font minus\fontdimen4\font
            \discretionary{\copy\Wrappedvisiblespacebox}{\Wrappedafterbreak}
            {\kern\fontdimen2\font}%
        }%
        
        % Allow breaks at special characters using \PYG... macros.
        \Wrappedbreaksatspecials
        % Breaks at punctuation characters . , ; ? ! and / need catcode=\active 	
        \OriginalVerbatim[#1,codes*=\Wrappedbreaksatpunct]%
    }
    \makeatother

    % Exact colors from NB
    \definecolor{incolor}{HTML}{303F9F}
    \definecolor{outcolor}{HTML}{D84315}
    \definecolor{cellborder}{HTML}{CFCFCF}
    \definecolor{cellbackground}{HTML}{F7F7F7}
    
    % prompt
    \makeatletter
    \newcommand{\boxspacing}{\kern\kvtcb@left@rule\kern\kvtcb@boxsep}
    \makeatother
    \newcommand{\prompt}[4]{
        {\ttfamily\llap{{\color{#2}[#3]:\hspace{3pt}#4}}\vspace{-\baselineskip}}
    }
    

    
    % Prevent overflowing lines due to hard-to-break entities
    \sloppy 
    % Setup hyperref package
    \hypersetup{
      breaklinks=true,  % so long urls are correctly broken across lines
      colorlinks=true,
      urlcolor=urlcolor,
      linkcolor=linkcolor,
      citecolor=citecolor,
      }
    % Slightly bigger margins than the latex defaults
    
    \geometry{verbose,tmargin=1in,bmargin=1in,lmargin=1in,rmargin=1in}
    
    

\begin{document}
    
    \maketitle
    
    

    
    \# Сведение задачи быстродействия к системе дифференциальных уравнений

    \hypertarget{ux444ux43eux440ux43cux443ux43bux438ux440ux43eux432ux43aux430-ux437ux430ux434ux430ux447ux438}{%
\subsection{Формулировка
задачи}\label{ux444ux43eux440ux43cux443ux43bux438ux440ux43eux432ux43aux430-ux437ux430ux434ux430ux447ux438}}

    Решается задача Лагранжа(47), с фиксированными концами:

\[\int_0^1 \ddot{x}^2 + \frac{48x}{1+ \alpha t^2} \rightarrow \operatorname{extr}\]
при \(x(1) = \dot{x}(0) = 0\) и \(\alpha \in \{ 0.0; 0.1; 1.0; 5.1\}\)

    \hypertarget{ux43dux435ux43eux431ux445ux43eux434ux438ux43cux44bux435-ux443ux441ux43bux43eux432ux438ux44f-ux43cux438ux43dux438ux43cux443ux43cux430}{%
\subsection{Необходимые условия
минимума}\label{ux43dux435ux43eux431ux445ux43eux434ux438ux43cux44bux435-ux443ux441ux43bux43eux432ux438ux44f-ux43cux438ux43dux438ux43cux443ux43cux430}}

    Для сокращения записей все громоздкие вычисления в отчете не приводятся.
Также автор позволяет себе неточность обозначений, не всегда выделяя
элементы оптимального решения \(\hat{x}\) по сравнению с неизвестными
\(x\).

Введем управление: \[y = \dot{x}, \ \ u = \dot{y}\]

Тогда задача переписывается:

\[\int_0^1 u^2 + \frac{48x}{1+ \alpha t^2} \rightarrow \operatorname{extr}\]
при \(x(1) = y(0) = 0\)

Запишем функцию Лагранжа:

\[ \mathcal{L} = \int_{0}^{1} L dt + l \]

\[L = \lambda_{0} \left( u^2 + \frac{48x}{1+ \alpha t^2} \right) + p_1 (\dot{x} - y) + p_2 (\dot{y} - u)\]

\[l = \lambda_1 x(1) + \lambda_2 y(0)\]

    И необходимые условия оптимальности:

\begin{enumerate}
\def\labelenumi{\arabic{enumi})}
\tightlist
\item
  система уравнений Эйлера:
\end{enumerate}

\[- \dot{p}_1 - \frac{48\lambda_0}{1+ \alpha t^2} = 0\]

\[-\dot{p}_2 - p_1 = 0\]

\begin{enumerate}
\def\labelenumi{\arabic{enumi})}
\setcounter{enumi}{1}
\tightlist
\item
  условия трансверсальности по x\_i:
\end{enumerate}

\[p_1(0) = 0; \ \ p_1(1) = -\lambda_1; \ \ \\
p_2(0) = \lambda_2; \ \ p_2(1) = 0; \ \ \]

\begin{enumerate}
\def\labelenumi{\arabic{enumi})}
\setcounter{enumi}{2}
\tightlist
\item
  условие минимума по \(u\)(слогаемые не зависящие от u пропущены):
\end{enumerate}

\[u = \operatorname{argmin}_{u} (\lambda_0 u^2 - p_2 u) = \frac{p_2}{2\lambda_0}\]

\begin{enumerate}
\def\labelenumi{\arabic{enumi})}
\setcounter{enumi}{3}
\tightlist
\item
  условие неотрицательности \(\lambda_0 \ge 0\)
\end{enumerate}

    \hypertarget{ux441ux438ux441ux442ux435ux43cux430-ux434ux438ux444ux444ux435ux440ux435ux43dux446ux438ux430ux43bux44cux43dux44bux445-ux443ux440ux430ux432ux43dux435ux43dux438ux439}{%
\subsection{Система дифференциальных
уравнений}\label{ux441ux438ux441ux442ux435ux43cux430-ux434ux438ux444ux444ux435ux440ux435ux43dux446ux438ux430ux43bux44cux43dux44bux445-ux443ux440ux430ux432ux43dux435ux43dux438ux439}}

    При \(\lambda_0=0\) получаем \(\lambda_1 = \lambda_2 = 0\),
следовательно этот случай можно откинуть. Положим \(\lambda_0 = 1/48\).
Получаем систему:

$$\begin{matrix}
\dot{x}  &=& y \\ 
\dot{y} &=& 48 p_2 \\
\dot{p_2} &=& - p_1\\
\dot{p_1}  &=& - \frac{1}{1+\alpha t^2} \\
\end{matrix}$$

С краевыми условиями \(x(1) = y(0) = p_1(0) = p_2(0) = 0\)

Дополняя систему недостающими начальными условиями:
\[x(0) = \alpha, \ \ p_2(0) = \beta\] Получаем задачу Коши(отказываясь
временно от граничных условий при \(t = 1\)),

    \hypertarget{ux430ux43dux430ux43bux438ux442ux438ux447ux435ux441ux43aux43eux435-ux440ux435ux448ux435ux43dux438ux435}{%
\subsection{Аналитическое
решение}\label{ux430ux43dux430ux43bux438ux442ux438ux447ux435ux441ux43aux43eux435-ux440ux435ux448ux435ux43dux438ux435}}

    \[\ddot{p_2} = \frac{1}{1+\alpha t^2}\] Рассматривая случай
\(\alpha>0\):
\[p_2 = \frac{t \arctan\left(\sqrt{\alpha} t\right)}{\sqrt{\alpha}} -\frac{\log \left(\alpha t^2+1\right)}{2 \alpha} +c_1 t+c_2\]

\(p_1(0) = \dot{p_2}(0) = 0 \Rightarrow\) \[c_1 = 0\]
\(p_2(1) = 0 \Rightarrow\)
\[c_1 = \frac{\arctan \left(\sqrt{\alpha}\right)}{\sqrt{\alpha}}-\frac{\log (\alpha+1)}{2 \alpha}\]

Итого:
\[p_2 = \frac{t \arctan\left(\sqrt{\alpha} t\right)  + \arctan \left(\sqrt{\alpha}\right)}{\sqrt{\alpha}} - 
\frac{\log \left(\alpha t^2+1\right) + \log (\alpha+1)}{2 \alpha} \]

Аналогично, для x, учитывая краевые условия получено(при помощи пакета
Wolfram Mathematica):
\[x(t) = \frac{4}{\alpha^2} \left(5 \alpha t^2+3 \alpha t^2 \log (a+1)-3 \alpha t^2 \log \left(\alpha t^2+1\right)+\log \left(\alpha t^2+1\right)+ \\
+2 \sqrt{\alpha} t \left(\alpha t^2-3\right) \tan ^{-1}\left(\sqrt{\alpha} t\right)-2 \sqrt{\alpha} \left(\alpha \left(3 t^2-2\right)-3\right) \tan ^{-1}\left(\sqrt{\alpha}\right)-5 \alpha -\log (\alpha+1)\right)\]

    Рассматривая случай \(\alpha = 0\):
\[x(t) = 2 \left(t^4-6 t^2+5\right)\]

    \hypertarget{ux432ux441ux43fux43eux43cux43eux433ux430ux442ux435ux43bux44cux43dux44bux439-ux43aux43eux434}{%
\section{Вспомогательный
код:}\label{ux432ux441ux43fux43eux43cux43eux433ux430ux442ux435ux43bux44cux43dux44bux439-ux43aux43eux434}}

    \begin{tcolorbox}[breakable, size=fbox, boxrule=1pt, pad at break*=1mm,colback=cellbackground, colframe=cellborder]
\prompt{In}{incolor}{1}{\boxspacing}
\begin{Verbatim}[commandchars=\\\{\}]
\PY{c+c1}{\PYZsh{}вспомогательный код}
\PY{k+kn}{from} \PY{n+nn}{rahmodule} \PY{k+kn}{import} \PY{o}{*}
\PY{k+kn}{import} \PY{n+nn}{numpy} \PY{k}{as} \PY{n+nn}{np}
\PY{k+kn}{import} \PY{n+nn}{os}
\PY{k+kn}{from} \PY{n+nn}{matplotlib} \PY{k+kn}{import} \PY{n}{pyplot} \PY{k}{as} \PY{n}{plt}
\PY{k+kn}{from} \PY{n+nn}{tqdm}\PY{n+nn}{.}\PY{n+nn}{notebook} \PY{k+kn}{import} \PY{n}{tqdm}
\end{Verbatim}
\end{tcolorbox}

    \begin{tcolorbox}[breakable, size=fbox, boxrule=1pt, pad at break*=1mm,colback=cellbackground, colframe=cellborder]
\prompt{In}{incolor}{2}{\boxspacing}
\begin{Verbatim}[commandchars=\\\{\}]
\PY{c+c1}{\PYZsh{}полученная система дифференциальных уравнений:}

\PY{k}{def} \PY{n+nf}{f1}\PY{p}{(}\PY{n}{t}\PY{p}{,} \PY{n}{X}\PY{p}{,} \PY{n}{a}\PY{o}{=}\PY{l+m+mi}{0}\PY{p}{)}\PY{p}{:}
    \PY{n}{x}\PY{p}{,} \PY{n}{y}\PY{p}{,} \PY{n}{p1}\PY{p}{,} \PY{n}{p2} \PY{o}{=} \PY{n}{X}
    \PY{c+c1}{\PYZsh{}print(y, u\PYZus{})}
    
    \PY{n}{xx1} \PY{o}{=} \PY{n}{y}
    \PY{n}{xx2} \PY{o}{=} \PY{l+m+mi}{48}\PY{o}{*}\PY{n}{p2}
    \PY{n}{pp2} \PY{o}{=} \PY{o}{\PYZhy{}}\PY{n}{p1}
    \PY{n}{pp1} \PY{o}{=} \PY{o}{\PYZhy{}}\PY{l+m+mi}{1}\PY{o}{/}\PY{p}{(}\PY{l+m+mi}{1}\PY{o}{+}\PY{n}{a}\PY{o}{*}\PY{n}{t}\PY{o}{*}\PY{o}{*}\PY{l+m+mi}{2}\PY{p}{)}
    \PY{k}{return} \PY{n}{np}\PY{o}{.}\PY{n}{array}\PY{p}{(}\PY{p}{[}\PY{n}{xx1}\PY{p}{,} \PY{n}{xx2}\PY{p}{,} \PY{n}{pp1}\PY{p}{,} \PY{n}{pp2}\PY{p}{]}\PY{p}{)}

\PY{c+c1}{\PYZsh{}аналитическое решение при \PYZbs{}alpha=0}
\PY{k}{def} \PY{n+nf}{analitic\PYZus{}func}\PY{p}{(}\PY{n}{t}\PY{p}{)}\PY{p}{:}
    \PY{k}{return} \PY{l+m+mi}{2}\PY{o}{*}\PY{p}{(}\PY{n}{t}\PY{o}{*}\PY{o}{*}\PY{l+m+mi}{4} \PY{o}{\PYZhy{}} \PY{l+m+mi}{6}\PY{o}{*}\PY{n}{t}\PY{o}{*}\PY{o}{*}\PY{l+m+mi}{2}\PY{o}{+}\PY{l+m+mi}{5}\PY{p}{)}
\end{Verbatim}
\end{tcolorbox}

    \begin{tcolorbox}[breakable, size=fbox, boxrule=1pt, pad at break*=1mm,colback=cellbackground, colframe=cellborder]
\prompt{In}{incolor}{3}{\boxspacing}
\begin{Verbatim}[commandchars=\\\{\}]
\PY{c+c1}{\PYZsh{}невязка}
\PY{k}{def} \PY{n+nf}{X}\PY{p}{(}\PY{n}{x0}\PY{p}{,} \PY{n}{p20}\PY{p}{,} \PY{n}{f}\PY{o}{=}\PY{n}{f1}\PY{p}{,} \PY{n}{T}\PY{o}{=}\PY{l+m+mi}{1}\PY{p}{,} \PY{n}{T\PYZus{}steps} \PY{o}{=} \PY{l+m+mi}{100}\PY{p}{,} \PY{n}{a} \PY{o}{=} \PY{l+m+mi}{0}\PY{p}{)}\PY{p}{:}
    \PY{n}{y0}\PY{p}{,} \PY{n}{p10} \PY{o}{=} \PY{l+m+mi}{0}\PY{p}{,} \PY{l+m+mi}{0}
    \PY{n}{Y} \PY{o}{=} \PY{n}{np}\PY{o}{.}\PY{n}{array}\PY{p}{(}\PY{p}{[}\PY{n}{x0}\PY{p}{,} \PY{n}{y0}\PY{p}{,} \PY{n}{p10}\PY{p}{,} \PY{n}{p20}\PY{p}{]}\PY{p}{)}
    \PY{n}{t\PYZus{}grid}\PY{p}{,} \PY{n}{ans}\PY{p}{,} \PY{n}{hs} \PY{o}{=} \PY{n}{solve}\PY{p}{(}\PY{n}{Y}\PY{p}{,} \PY{n}{T}\PY{o}{=}\PY{n}{T}\PY{p}{,} \PY{n}{T\PYZus{}steps}\PY{o}{=}\PY{n}{T\PYZus{}steps}\PY{p}{,} \PY{n}{a}\PY{o}{=}\PY{n}{a}\PY{p}{,} \PY{n}{f}\PY{o}{=}\PY{n}{f}\PY{p}{)}
    \PY{n}{X\PYZus{}} \PY{o}{=} \PY{p}{[}\PY{n}{ans}\PY{p}{[}\PY{o}{\PYZhy{}}\PY{l+m+mi}{1}\PY{p}{]}\PY{p}{[}\PY{l+m+mi}{0}\PY{p}{]}\PY{p}{,} \PY{n}{ans}\PY{p}{[}\PY{o}{\PYZhy{}}\PY{l+m+mi}{1}\PY{p}{]}\PY{p}{[}\PY{l+m+mi}{3}\PY{p}{]}\PY{p}{]}
    
    \PY{c+c1}{\PYZsh{}return np.append(X\PYZus{}, p1*x1*us[\PYZhy{}1] + 1/6*(x2 \PYZhy{} 4*x1)*(p1 \PYZhy{} p2))}
    \PY{k}{return} \PY{n}{np}\PY{o}{.}\PY{n}{array}\PY{p}{(}\PY{n}{X\PYZus{}}\PY{p}{)}
\end{Verbatim}
\end{tcolorbox}

    \begin{tcolorbox}[breakable, size=fbox, boxrule=1pt, pad at break*=1mm,colback=cellbackground, colframe=cellborder]
\prompt{In}{incolor}{4}{\boxspacing}
\begin{Verbatim}[commandchars=\\\{\}]
\PY{k}{def} \PY{n+nf}{save}\PY{p}{(}\PY{n}{point}\PY{p}{,} \PY{n}{ans}\PY{p}{,} \PY{n}{t\PYZus{}grid}\PY{p}{,} \PY{n}{alpha}\PY{p}{,} \PY{n}{pth}\PY{p}{)}\PY{p}{:}
    \PY{k}{if} \PY{o+ow}{not} \PY{n}{os}\PY{o}{.}\PY{n}{path}\PY{o}{.}\PY{n}{isdir}\PY{p}{(}\PY{n}{pth}\PY{p}{)}\PY{p}{:}
        \PY{n}{os}\PY{o}{.}\PY{n}{mkdir}\PY{p}{(}\PY{n}{pth}\PY{p}{)}
    
    \PY{k}{with} \PY{n+nb}{open}\PY{p}{(}\PY{l+s+sa}{f}\PY{l+s+s2}{\PYZdq{}}\PY{l+s+s2}{./}\PY{l+s+si}{\PYZob{}}\PY{n}{pth}\PY{l+s+si}{\PYZcb{}}\PY{l+s+s2}{/params.txt}\PY{l+s+s2}{\PYZdq{}}\PY{p}{,} \PY{l+s+s1}{\PYZsq{}}\PY{l+s+s1}{w}\PY{l+s+s1}{\PYZsq{}}\PY{p}{)} \PY{k}{as} \PY{n}{X\PYZus{}\PYZus{}}\PY{p}{:}
        \PY{n}{X\PYZus{}\PYZus{}}\PY{o}{.}\PY{n}{write}\PY{p}{(}\PY{l+s+sa}{f}\PY{l+s+s2}{\PYZdq{}}\PY{l+s+s2}{alpha: }\PY{l+s+si}{\PYZob{}}\PY{n}{alpha}\PY{l+s+si}{\PYZcb{}}\PY{l+s+s2}{, T points: }\PY{l+s+si}{\PYZob{}}\PY{n+nb}{len}\PY{p}{(}\PY{n}{t\PYZus{}grid}\PY{p}{)}\PY{l+s+si}{\PYZcb{}}\PY{l+s+se}{\PYZbs{}n}\PY{l+s+s2}{\PYZdq{}}\PY{p}{)}
        \PY{n}{X\PYZus{}\PYZus{}}\PY{o}{.}\PY{n}{write}\PY{p}{(}\PY{l+s+sa}{f}\PY{l+s+s2}{\PYZdq{}}\PY{l+s+s2}{x0: }\PY{l+s+si}{\PYZob{}}\PY{n}{point}\PY{p}{[}\PY{l+m+mi}{0}\PY{p}{]}\PY{l+s+si}{\PYZcb{}}\PY{l+s+s2}{, p20: }\PY{l+s+si}{\PYZob{}}\PY{n}{point}\PY{p}{[}\PY{l+m+mi}{1}\PY{p}{]}\PY{l+s+si}{\PYZcb{}}\PY{l+s+se}{\PYZbs{}n}\PY{l+s+s2}{\PYZdq{}}\PY{p}{)}
        \PY{n}{X\PYZus{}\PYZus{}}\PY{o}{.}\PY{n}{write}\PY{p}{(}\PY{l+s+sa}{f}\PY{l+s+s2}{\PYZdq{}}\PY{l+s+s2}{Discrepancy: }\PY{l+s+si}{\PYZob{}}\PY{p}{[}\PY{n}{ans}\PY{p}{[}\PY{o}{\PYZhy{}}\PY{l+m+mi}{1}\PY{p}{]}\PY{p}{[}\PY{l+m+mi}{0}\PY{p}{]}\PY{p}{,} \PY{n}{ans}\PY{p}{[}\PY{o}{\PYZhy{}}\PY{l+m+mi}{1}\PY{p}{]}\PY{p}{[}\PY{l+m+mi}{3}\PY{p}{]}\PY{p}{]}\PY{l+s+si}{\PYZcb{}}\PY{l+s+s2}{\PYZdq{}}\PY{p}{)}
        
    
    \PY{k}{with} \PY{n+nb}{open}\PY{p}{(}\PY{l+s+sa}{f}\PY{l+s+s2}{\PYZdq{}}\PY{l+s+s2}{./}\PY{l+s+si}{\PYZob{}}\PY{n}{pth}\PY{l+s+si}{\PYZcb{}}\PY{l+s+s2}{/trajectory.txt}\PY{l+s+s2}{\PYZdq{}}\PY{p}{,} \PY{l+s+s1}{\PYZsq{}}\PY{l+s+s1}{w}\PY{l+s+s1}{\PYZsq{}}\PY{p}{)} \PY{k}{as} \PY{n}{X\PYZus{}\PYZus{}}\PY{p}{:}
            \PY{k}{for} \PY{n}{i} \PY{o+ow}{in} \PY{n+nb}{range}\PY{p}{(}\PY{n+nb}{len}\PY{p}{(}\PY{n}{t\PYZus{}grid}\PY{p}{)}\PY{p}{)}\PY{p}{:}  
                \PY{n}{X\PYZus{}\PYZus{}}\PY{o}{.}\PY{n}{write}\PY{p}{(}\PY{l+s+sa}{f}\PY{l+s+s2}{\PYZdq{}}\PY{l+s+si}{\PYZob{}}\PY{n}{t\PYZus{}grid}\PY{p}{[}\PY{n}{i}\PY{p}{]}\PY{l+s+si}{\PYZcb{}}\PY{l+s+s2}{, }\PY{l+s+si}{\PYZob{}}\PY{n}{ans}\PY{p}{[}\PY{n}{i}\PY{p}{]}\PY{p}{[}\PY{l+m+mi}{0}\PY{p}{]}\PY{l+s+si}{\PYZcb{}}\PY{l+s+s2}{, }\PY{l+s+si}{\PYZob{}}\PY{n}{ans}\PY{p}{[}\PY{n}{i}\PY{p}{]}\PY{p}{[}\PY{l+m+mi}{1}\PY{p}{]}\PY{l+s+si}{\PYZcb{}}\PY{l+s+s2}{, }\PY{l+s+si}{\PYZob{}}\PY{n}{ans}\PY{p}{[}\PY{n}{i}\PY{p}{]}\PY{p}{[}\PY{l+m+mi}{2}\PY{p}{]}\PY{l+s+si}{\PYZcb{}}\PY{l+s+s2}{, }\PY{l+s+si}{\PYZob{}}\PY{n}{ans}\PY{p}{[}\PY{n}{i}\PY{p}{]}\PY{p}{[}\PY{l+m+mi}{3}\PY{p}{]}\PY{l+s+si}{\PYZcb{}}\PY{l+s+s2}{ }\PY{l+s+se}{\PYZbs{}n}\PY{l+s+s2}{\PYZdq{}}\PY{p}{)}
    
\end{Verbatim}
\end{tcolorbox}

    \hypertarget{ux440ux435ux448ux435ux43dux438ux435-ux43fux440ux438-alpha-0-ux441ux440ux430ux432ux43dux435ux43dux438ux435-ux441-ux430ux43dux430ux43bux438ux442ux438ux447ux435ux441ux43aux438ux43c.}{%
\section{\texorpdfstring{Решение при \(\alpha = 0\), сравнение с
аналитическим.}{Решение при \textbackslash alpha = 0, сравнение с аналитическим.}}\label{ux440ux435ux448ux435ux43dux438ux435-ux43fux440ux438-alpha-0-ux441ux440ux430ux432ux43dux435ux43dux438ux435-ux441-ux430ux43dux430ux43bux438ux442ux438ux447ux435ux441ux43aux438ux43c.}}

    \begin{tcolorbox}[breakable, size=fbox, boxrule=1pt, pad at break*=1mm,colback=cellbackground, colframe=cellborder]
\prompt{In}{incolor}{5}{\boxspacing}
\begin{Verbatim}[commandchars=\\\{\}]
\PY{n}{a} \PY{o}{=} \PY{l+m+mi}{0}
\PY{n}{point} \PY{o}{=} \PY{n}{newtonian\PYZus{}finder}\PY{p}{(}\PY{n}{X}\PY{p}{,} \PY{p}{[}\PY{l+m+mi}{0}\PY{p}{,} \PY{l+m+mi}{0}\PY{p}{]}\PY{p}{,} \PY{n}{dim}\PY{o}{=}\PY{l+m+mi}{2}\PY{p}{,} \PY{n}{a}\PY{o}{=}\PY{n}{a}\PY{p}{)}
\PY{n}{y0} \PY{o}{=} \PY{p}{[}\PY{n}{point}\PY{p}{[}\PY{l+m+mi}{0}\PY{p}{]}\PY{p}{,} \PY{l+m+mi}{0}\PY{p}{,} \PY{l+m+mi}{0}\PY{p}{,} \PY{n}{point}\PY{p}{[}\PY{l+m+mi}{1}\PY{p}{]}\PY{p}{]}
\PY{n}{t\PYZus{}grid}\PY{p}{,} \PY{n}{ans}\PY{p}{,} \PY{n}{hs} \PY{o}{=} \PY{n}{solve}\PY{p}{(}\PY{n}{y0}\PY{p}{,} \PY{l+m+mi}{1}\PY{p}{,} \PY{n}{T\PYZus{}steps} \PY{o}{=} \PY{l+m+mi}{100}\PY{p}{,} \PY{n}{a} \PY{o}{=} \PY{n}{a}\PY{p}{,} \PY{n}{f} \PY{o}{=} \PY{n}{f1}\PY{p}{)}
\end{Verbatim}
\end{tcolorbox}

    \begin{tcolorbox}[breakable, size=fbox, boxrule=1pt, pad at break*=1mm,colback=cellbackground, colframe=cellborder]
\prompt{In}{incolor}{6}{\boxspacing}
\begin{Verbatim}[commandchars=\\\{\}]
\PY{n}{AN\PYZus{}SOL} \PY{o}{=} \PY{p}{[}\PY{p}{]}
\PY{k}{for} \PY{n}{t} \PY{o+ow}{in} \PY{n}{np}\PY{o}{.}\PY{n}{arange}\PY{p}{(}\PY{l+m+mi}{0}\PY{p}{,} \PY{l+m+mi}{1}\PY{o}{+}\PY{l+m+mf}{0.01}\PY{p}{,} \PY{l+m+mf}{0.01}\PY{p}{)}\PY{p}{:}
    \PY{n}{AN\PYZus{}SOL}\PY{o}{.}\PY{n}{append}\PY{p}{(}\PY{n}{analitic\PYZus{}func}\PY{p}{(}\PY{n}{t}\PY{p}{)}\PY{p}{)}
\PY{n}{AN\PYZus{}SOL} \PY{o}{=} \PY{n}{np}\PY{o}{.}\PY{n}{array}\PY{p}{(}\PY{n}{AN\PYZus{}SOL}\PY{p}{)}
\end{Verbatim}
\end{tcolorbox}

    Аналитическое и численное решение на одном графике:

    \begin{tcolorbox}[breakable, size=fbox, boxrule=1pt, pad at break*=1mm,colback=cellbackground, colframe=cellborder]
\prompt{In}{incolor}{7}{\boxspacing}
\begin{Verbatim}[commandchars=\\\{\}]
\PY{n}{plt}\PY{o}{.}\PY{n}{plot}\PY{p}{(}\PY{n}{t\PYZus{}grid}\PY{p}{,} \PY{n}{np}\PY{o}{.}\PY{n}{array}\PY{p}{(}\PY{p}{[}\PY{n}{np}\PY{o}{.}\PY{n}{array}\PY{p}{(}\PY{n}{ans}\PY{p}{)}\PY{o}{.}\PY{n}{T}\PY{p}{[}\PY{l+m+mi}{0}\PY{p}{]}\PY{p}{,} \PY{n}{AN\PYZus{}SOL}\PY{p}{]}\PY{p}{)}\PY{o}{.}\PY{n}{T}\PY{p}{)}
\end{Verbatim}
\end{tcolorbox}

            \begin{tcolorbox}[breakable, size=fbox, boxrule=.5pt, pad at break*=1mm, opacityfill=0]
\prompt{Out}{outcolor}{7}{\boxspacing}
\begin{Verbatim}[commandchars=\\\{\}]
[<matplotlib.lines.Line2D at 0x140a3e8d4c8>,
 <matplotlib.lines.Line2D at 0x140a1e351c8>]
\end{Verbatim}
\end{tcolorbox}
        
    \begin{center}
    \adjustimage{max size={0.9\linewidth}{0.9\paperheight}}{output_21_1.png}
    \end{center}
    { \hspace*{\fill} \\}
    
    Разница аналитического и точного решения:

    \begin{tcolorbox}[breakable, size=fbox, boxrule=1pt, pad at break*=1mm,colback=cellbackground, colframe=cellborder]
\prompt{In}{incolor}{8}{\boxspacing}
\begin{Verbatim}[commandchars=\\\{\}]
\PY{n}{plt}\PY{o}{.}\PY{n}{plot}\PY{p}{(}\PY{n}{t\PYZus{}grid}\PY{p}{,} \PY{n}{np}\PY{o}{.}\PY{n}{array}\PY{p}{(}\PY{n}{ans}\PY{p}{)}\PY{o}{.}\PY{n}{T}\PY{p}{[}\PY{l+m+mi}{0}\PY{p}{]} \PY{o}{\PYZhy{}} \PY{n}{AN\PYZus{}SOL}\PY{p}{)}
\end{Verbatim}
\end{tcolorbox}

            \begin{tcolorbox}[breakable, size=fbox, boxrule=.5pt, pad at break*=1mm, opacityfill=0]
\prompt{Out}{outcolor}{8}{\boxspacing}
\begin{Verbatim}[commandchars=\\\{\}]
[<matplotlib.lines.Line2D at 0x140a3fb0608>]
\end{Verbatim}
\end{tcolorbox}
        
    \begin{center}
    \adjustimage{max size={0.9\linewidth}{0.9\paperheight}}{output_23_1.png}
    \end{center}
    { \hspace*{\fill} \\}
    
    \begin{tcolorbox}[breakable, size=fbox, boxrule=1pt, pad at break*=1mm,colback=cellbackground, colframe=cellborder]
\prompt{In}{incolor}{9}{\boxspacing}
\begin{Verbatim}[commandchars=\\\{\}]
\PY{n}{save}\PY{p}{(}\PY{n}{point} \PY{o}{=} \PY{n}{point}\PY{p}{,} \PY{n}{ans}\PY{o}{=}\PY{n}{ans}\PY{p}{,} \PY{n}{t\PYZus{}grid}\PY{o}{=}\PY{n}{t\PYZus{}grid}\PY{p}{,} \PY{n}{alpha}\PY{o}{=}\PY{n}{a}\PY{p}{,} \PY{n}{pth}\PY{o}{=}\PY{l+s+s2}{\PYZdq{}}\PY{l+s+s2}{00}\PY{l+s+s2}{\PYZdq{}}\PY{p}{)}
\end{Verbatim}
\end{tcolorbox}

    \hypertarget{ux43eux441ux442ux430ux43bux44cux43dux44bux435-alpha}{%
\section{\texorpdfstring{Остальные
\(\alpha\)}{Остальные \textbackslash alpha}}\label{ux43eux441ux442ux430ux43bux44cux43dux44bux435-alpha}}

    \begin{tcolorbox}[breakable, size=fbox, boxrule=1pt, pad at break*=1mm,colback=cellbackground, colframe=cellborder]
\prompt{In}{incolor}{10}{\boxspacing}
\begin{Verbatim}[commandchars=\\\{\}]
\PY{k}{for} \PY{n}{a} \PY{o+ow}{in} \PY{p}{[}\PY{l+m+mf}{0.1}\PY{p}{,} \PY{l+m+mi}{1}\PY{p}{,} \PY{l+m+mf}{5.1}\PY{p}{]}\PY{p}{:}
    \PY{n}{point} \PY{o}{=} \PY{n}{newtonian\PYZus{}finder}\PY{p}{(}\PY{n}{X}\PY{p}{,} \PY{p}{[}\PY{l+m+mi}{0}\PY{p}{,} \PY{l+m+mi}{0}\PY{p}{]}\PY{p}{,} \PY{n}{dim}\PY{o}{=}\PY{l+m+mi}{2}\PY{p}{,} \PY{n}{a}\PY{o}{=}\PY{n}{a}\PY{p}{)}
    \PY{n}{y0} \PY{o}{=} \PY{p}{[}\PY{n}{point}\PY{p}{[}\PY{l+m+mi}{0}\PY{p}{]}\PY{p}{,} \PY{l+m+mi}{0}\PY{p}{,} \PY{l+m+mi}{0}\PY{p}{,} \PY{n}{point}\PY{p}{[}\PY{l+m+mi}{1}\PY{p}{]}\PY{p}{]}
    \PY{n+nb}{print}\PY{p}{(}\PY{l+s+sa}{f}\PY{l+s+s2}{\PYZdq{}}\PY{l+s+s2}{При \PYZdl{}}\PY{l+s+se}{\PYZbs{}\PYZbs{}}\PY{l+s+s2}{alpha = }\PY{l+s+si}{\PYZob{}}\PY{n}{a}\PY{l+s+si}{\PYZcb{}}\PY{l+s+s2}{\PYZdl{}, получены начальные условия: }\PY{l+s+se}{\PYZbs{}n}\PY{l+s+s2}{ \PYZdl{}x(0) = }\PY{l+s+si}{\PYZob{}}\PY{n}{point}\PY{p}{[}\PY{l+m+mi}{0}\PY{p}{]}\PY{l+s+si}{\PYZcb{}}\PY{l+s+s2}{\PYZdl{}, \PYZdl{}p\PYZus{}1 (0) = }\PY{l+s+si}{\PYZob{}}\PY{n}{point}\PY{p}{[}\PY{l+m+mi}{0}\PY{p}{]}\PY{l+s+si}{\PYZcb{}}\PY{l+s+s2}{\PYZdl{}}\PY{l+s+s2}{\PYZdq{}}\PY{p}{)}
    \PY{n+nb}{print}\PY{p}{(}\PY{l+s+sa}{f}\PY{l+s+s2}{\PYZdq{}}\PY{l+s+s2}{Невязка: }\PY{l+s+si}{\PYZob{}}\PY{p}{[}\PY{n}{ans}\PY{p}{[}\PY{o}{\PYZhy{}}\PY{l+m+mi}{1}\PY{p}{]}\PY{p}{[}\PY{l+m+mi}{0}\PY{p}{]}\PY{p}{,} \PY{n}{ans}\PY{p}{[}\PY{o}{\PYZhy{}}\PY{l+m+mi}{1}\PY{p}{]}\PY{p}{[}\PY{l+m+mi}{3}\PY{p}{]}\PY{p}{]}\PY{l+s+si}{\PYZcb{}}\PY{l+s+s2}{\PYZdq{}}\PY{p}{)} 
    \PY{n}{t\PYZus{}grid}\PY{p}{,} \PY{n}{ans}\PY{p}{,} \PY{n}{hs} \PY{o}{=} \PY{n}{solve}\PY{p}{(}\PY{n}{y0}\PY{p}{,} \PY{l+m+mi}{1}\PY{p}{,} \PY{n}{T\PYZus{}steps} \PY{o}{=} \PY{l+m+mi}{100}\PY{p}{,} \PY{n}{a} \PY{o}{=} \PY{n}{a}\PY{p}{,} \PY{n}{f} \PY{o}{=} \PY{n}{f1}\PY{p}{)}
    \PY{n+nb}{print}\PY{p}{(}\PY{l+s+s2}{\PYZdq{}}\PY{l+s+s2}{График численного решения:}\PY{l+s+s2}{\PYZdq{}}\PY{p}{)}
    \PY{n}{plt}\PY{o}{.}\PY{n}{plot}\PY{p}{(}\PY{n}{t\PYZus{}grid}\PY{p}{,} \PY{n}{np}\PY{o}{.}\PY{n}{array}\PY{p}{(}\PY{n}{ans}\PY{p}{)}\PY{o}{.}\PY{n}{T}\PY{p}{[}\PY{l+m+mi}{0}\PY{p}{]}\PY{p}{)}
    \PY{n}{plt}\PY{o}{.}\PY{n}{show}\PY{p}{(}\PY{p}{)}
    \PY{n}{save}\PY{p}{(}\PY{n}{point} \PY{o}{=} \PY{n}{point}\PY{p}{,} \PY{n}{ans}\PY{o}{=}\PY{n}{ans}\PY{p}{,} \PY{n}{t\PYZus{}grid}\PY{o}{=}\PY{n}{t\PYZus{}grid}\PY{p}{,} \PY{n}{alpha}\PY{o}{=}\PY{n}{a}\PY{p}{,} \PY{n}{pth}\PY{o}{=}\PY{l+s+sa}{f}\PY{l+s+s2}{\PYZdq{}}\PY{l+s+si}{\PYZob{}}\PY{n+nb}{int}\PY{p}{(}\PY{n}{a}\PY{o}{*}\PY{l+m+mi}{10}\PY{p}{)}\PY{l+s+si}{\PYZcb{}}\PY{l+s+s2}{\PYZdq{}}\PY{p}{)}
\end{Verbatim}
\end{tcolorbox}

    \begin{Verbatim}[commandchars=\\\{\}]
При \$\textbackslash{}alpha = 0.1\$, получены начальные условия:
 \$x(0) = 9.82065315702208\$, \$p\_1 (0) = 9.82065315702208\$
Невязка: [-5.3520438081378074e-11, 3.275364354737853e-12]
График численного решения:
    \end{Verbatim}

    \begin{center}
    \adjustimage{max size={0.9\linewidth}{0.9\paperheight}}{output_26_1.png}
    \end{center}
    { \hspace*{\fill} \\}
    
    \begin{Verbatim}[commandchars=\\\{\}]
При \$\textbackslash{}alpha = 1\$, получены начальные условия:
 \$x(0) = 8.643337811787305\$, \$p\_1 (0) = 8.643337811787305\$
Невязка: [-8.530476325319114e-11, 4.923145224822179e-13]
График численного решения:
    \end{Verbatim}

    \begin{center}
    \adjustimage{max size={0.9\linewidth}{0.9\paperheight}}{output_26_3.png}
    \end{center}
    { \hspace*{\fill} \\}
    
    \begin{Verbatim}[commandchars=\\\{\}]
При \$\textbackslash{}alpha = 5.1\$, получены начальные условия:
 \$x(0) = 6.380489685459042\$, \$p\_1 (0) = 6.380489685459042\$
Невязка: [6.9111383282915995e-15, -3.122502256758253e-17]
График численного решения:
    \end{Verbatim}

    \begin{center}
    \adjustimage{max size={0.9\linewidth}{0.9\paperheight}}{output_26_5.png}
    \end{center}
    { \hspace*{\fill} \\}
    
    \begin{tcolorbox}[breakable, size=fbox, boxrule=1pt, pad at break*=1mm,colback=cellbackground, colframe=cellborder]
\prompt{In}{incolor}{ }{\boxspacing}
\begin{Verbatim}[commandchars=\\\{\}]

\end{Verbatim}
\end{tcolorbox}


    % Add a bibliography block to the postdoc
    
    
    
\end{document}
